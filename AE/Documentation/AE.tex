\documentclass[12pt]{article}
\usepackage[
singlelinecheck=false % The magic that stops caption centering
]{caption}
\usepackage{color}
\usepackage{graphicx}
\usepackage{grffile}	% makes graphicx less fussy about file names
\usepackage{longtable}
\setlength{\LTleft}{0pt} % put longtables on the left

% Trick to hide unused columns in tables
\usepackage{array}
\newcolumntype{H}{@{}>{\lrbox0}l<{\endlrbox}}


% Flush graphics before every new section
\usepackage{placeins}
\let\oldsection\section
\renewcommand{\section}{\FloatBarrier\oldsection} 
\let\oldsubsection\subsection
\renewcommand{\subsection}{\FloatBarrier\oldsubsection} 
\let\oldsubsubsection\subsubsection
\renewcommand{\subsubsection}{\FloatBarrier\oldsubsubsection} 

% Space out paragraphs, don't indent
\setlength{\parindent}{0.0in}
\setlength{\parskip}{0.1in}

% Macro for including schematic pages
\newcommand{\schempage}[1]{
   \begin{figure}[ht!]
   \centerline{\includegraphics[width=1.3\textwidth,angle=0,keepaspectratio=true]{#1.pdf}}
    \caption{#1}
    \label{#1}
    \end{figure}
}


\author{John P. Doty}
\title{TESS Focal Plane Electronics Manual}
\date{\input{date.txt}}

\begin{document}
\begin{titlepage}
\maketitle
\begin{center}
Very Preliminary Edition

Commit \input{stamp.txt}
\end{center}
\end{titlepage} 

\section{Introduction}
\section{Video Board}
\subsection{Building blocks}
\schempage{Chain.1}
\schempage{Chain.2}
\schempage{DrainRegulator}
\schempage{PerChip.1}
\schempage{PerChip.2}
\schempage{PerChip.3}
\schempage{PerChip.4}
\schempage{PerChip.5}
\schempage{PerChip.6}
\schempage{PerChip.7}
\schempage{PerChip.8}
\subsection{Video Board Top Level}
\schempage{Video.1}
\schempage{Video.2}
\schempage{Video.3}
\schempage{Video.4}
\schempage{Video.5}
\schempage{Video.6}
\schempage{Video.7}
\schempage{Video.8}
\subsection{Video Board Connectors}

\begin{longtable}{|l|l|l|l|}
\caption{Flexprint Connector} \label{J1} \\
\hline
Connector & Pin & Net & Signal \\
\hline \endfirsthead
\caption{Flexprint Connector (continued)} \\
\hline 
Connector & Pin & Net & Signal \\
\hline
\endhead
\hline \endfoot
\input{CCD-51pin.tex}
\end{longtable}

\begin{table}[ht!]
\caption{Temperature Connector}
\begin{tabular}{|l|l|lH|} %Comment column hidden for now
\hline
Connector & Pin & Net & Comment \\
\hline
\input{TempConn.tex}
\hline
\end{tabular}
\label{J5}
\end{table}

\section{Video Board to Interface Board Interconnection}

\begin{longtable}{|l|l|l|l|}
\caption{Stack Connectors} \label{stack} \\
\hline
Connector & Pin & Net & Comment \\
\hline \endfirsthead
\caption{Stack Connectors (continued)} \\
\hline 
Connector & Pin & Net & Comment \\
\hline
\endhead
\hline \endfoot
\input{video_interconnect.tex}
\end{longtable}


\section{Interface Board}
\subsection{Building blocks}
\subsubsection{Drivers for high capacitance (parallel) clocks}
\schempage{Booster}
\schempage{ParallelPair}
\schempage{ParallelReg}
\subsubsection{Drivers for low capacitance clocks}
\schempage{SerialDriver}
\schempage{SerialRegulator}
\subsubsection{Clock drivers for one CCD}
\schempage{DriverSet.1}
\schempage{DriverSet.2}
\schempage{DriverSet.3}
\schempage{DriverSet.4}
\subsubsection{Power conditioning}
\schempage{Pump}
\schempage{ArtixPower}
\subsection{Interface Board Top Level}
\schempage{Interface.1}
\schempage{Interface.2}
\schempage{Interface.3}
\schempage{Interface.4}
\schempage{Interface.5}
\schempage{Interface.6}
\schempage{Interface.7}
\schempage{Interface.8}
\schempage{Interface.9}
\schempage{Interface.10}
\subsection{Artix FPGA}

\begin{longtable}{|Hl|l|l|} %Hide the refdes field
\caption{Artix FPGA Connections} \label{Artix} \\
\hline
Connector & Pin & Net & Signal \\
\hline \endfirsthead
\caption{Artix FPGA Connections (continued)} \\
\hline 
Connector & Pin & Net & Signal \\
\hline
\endhead
\hline \endfoot
\input{Interface.U4.tex}
\end{longtable}

\subsection{Interface Board Connectors}

\subsubsection{FPGA Test Header}
\begin{table}[ht!]
\caption{FPGA Test Header}
\begin{tabular}{|l|l|lH|} %Comment column hidden for now
\hline
Connector & Pin & Net & Comment \\
\hline
\input{Interface.J4.tex}
\hline
\end{tabular}
\label{IJ4}
\end{table}

\section{Housekeeping Channels and DAC-controlled operating parameters}

While the implementation details differ, the housekeeping channels and the DAC-controlled parameters share a common addressing scheme. An address is seven bits. All seven are provided to the multiplexors and their selection logic as HKA[6:0]. The most significant four bits DCS[3:0] drive the DAC selection logic: the least significant three bits are part of the serial command that sets a DAC. In the following tables CC represents two bits selecting the CCD in offset binary (00$\Rightarrow$CCD1, 01$\Rightarrow$CCD2, 10$\Rightarrow$CCD3, 11$\Rightarrow$CCD4).

The control ranges often go outside the actual range allowed for the parameters, which depend on circuit details and power supply voltages. \textcolor{red}{I will document these limits in the future.} Control is sometimes relative to another parameter. If the control range is not given, the parameter is not under DAC control.
\subsection{Bias Group}
\begin{table}[ht!]
\caption{Bias Group}
\begin{tabular}{|l|l|l|l|}
\hline
\multicolumn{4}{|l|}{Address 0CCXXXX} \\
\hline
XXXX & Housekeeping Signal & Scale & Control Range \\
\hline
0000 & Output Gate & $\pm$16.5V & -14.9V, 5.0V \\
0001 & Input Gate 1 & $\pm$16.5V   & -14.9V, 5.0V \\
0010 & Input Gate 2 & $\pm$16.5V   & -14.9V, 5.0V \\
0011 & Scupper & $\pm$16.5V  & 0, 10.2V \\
0100 & Reset Drain & $\pm$16.5V & 0, 18.2V \\
0101 & Backside & $\pm$16.5V & 0, 5.0V \\
0110 & Substrate & $\pm$82V  & 0, -76.4 \\
0111 & Board Temperature & $\pm$360K &\\
1000 & Output Drain A & $\pm$27.3V & 0, 10.0V$\dagger$ \\
1001 & Output Drain B & $\pm$27.3V& 0, 10.0V$\dagger$ \\
1010 & Output Drain C & $\pm$27.3V& 0, 10.0V$\dagger$ \\
1011 & Output Drain D & $\pm$27.3V&  0, 10.0V$\dagger$ \\
1100 & Output Source A & $\pm$27.3V &\\
1100 & Output Source B & $\pm$27.3V &\\
1100 & Output Source C & $\pm$27.3V &\\
1100 & Output Source D & $\pm$27.3V &\\
\hline
\end{tabular}
\vspace{5pt}

$\dagger$ Relative to Reset Drain for the specified chip
\label{biastab}
\end{table}

\textcolor{red}{Clearly, the control ranges of the Bias Group (Table \ref{biastab}) need review and adjustment.}
\subsection{Clock Driver Group}
\begin{table}[ht!]
\caption{Clock Driver Group}
\begin{tabular}{|l|l|l|l|}
\hline
\multicolumn{4}{|l|}{Address 10CCXXX} \\
\hline
XXX & Housekeeping Signal & Scale  & Control Range \\
\hline
000 & Parallel High & $\pm$16.5V & 0, 18.1V$\dagger$ \\
001 &Parallel Low &$\pm$16.5V & 0, -13.2V \\
010 & Serial High &$\pm$16.5V & 14.9V, -14.9V \\
011 & Serial Low &$\pm$16.5V & 14.9V, -14.9V \\
100 & Reset High &$\pm$16.5V & 14.9V, -14.9V \\
101 &Reset Low &$\pm$16.5V & 14.9V, -14.9V \\
110 &Input Diode High &$\pm$16.5V & 0, +15.0V \\
111 & Input Diode Low &$\pm$16.5V & 0, +15.0V \\
\hline
\end{tabular}
\vspace{5pt}

$\dagger$ Relative to Parallel Low for the specified chip
\label{clocktab}
\end{table}
\subsection{Heater Group}
\begin{table}[ht!]
\caption{Heater Group}
\begin{tabular}{|l|l|l|l|}
\hline
\multicolumn{4}{|l|}{Address 1100000} \\
\hline
XXX & Housekeeping Signal & Scale &  Control Range \\
\hline
1100000 & Heater Current & $\pm$418mA & 0, 12.5V$\dagger$ \\
\hline
\end{tabular}
\vspace{5pt}

$\dagger$ Control range in mA depends on the external heater resistance
\label{heattab}
\end{table}
\subsection{Interface Group}
\begin{table}[ht!]
\caption{Interface Group}
\begin{tabular}{|l|l|l|}
\hline
\multicolumn{3}{|l|}{Address 1101XXX} \\
\hline
XXX & Housekeeping Signal & Scale  \\
\hline
000 & Board Temperature & $\pm$360K\\
001 & +15 & $\pm$16.5V\\
010 & +5 & $\pm$16.5V\\
011 & -12 & $\pm$16.5V\\
100 & +3.3F & $\pm$16.5V\\
101 & +2.5F & $\pm$16.5V\\
110 & +1.8F & $\pm$16.5V\\
111 & +1F & $\pm$16.5V\\
\hline
\end{tabular}
\label{inttab}
\end{table}
\subsection{Thermal Group}
\begin{table}[ht!]
\caption{Thermal Group}
\begin{tabular}{|l|l|l|}
\hline
\multicolumn{3}{|l|}{Address 111XXXX} \\
\hline
XXX & Housekeeping Signal & Scale  \\
\hline
0000 & Pt1000 sensor 1 & -125C, +130C\\
0001 & Pt1000 sensor 2 & -125C, +130C\\
0010 & Pt1000 sensor 3 & -125C, +130C\\
0011 & Pt1000 sensor 4 & -125C, +130C\\
0100 & Pt1000 sensor 5 & -125C, +130C\\
0101 & Pt1000 sensor 6 & -125C, +130C\\
0110 & Pt1000 sensor 7 & -125C, +130C\\
0111 & Pt1000 sensor 8 & -125C, +130C\\
1000 & Pt1000 sensor 9 & -125C, +130C\\
1001 & Pt1000 sensor 10 & -125C, +130C\\
1010 & Pt1000 sensor 11 & -125C, +130C\\
1011 & Pt1000 sensor 12 & -125C, +130C\\
1100 & AlCu sensor CCD1 & -150C, +40C\\
1101 & AlCu sensor CCD2 & -150C, +40C\\
1110 & AlCu sensor CCD3 & -150C, +40C\\
1111 & AlCu sensor CCD4 & -150C, +40C\\
\hline
\end{tabular}
\label{ttab}
\end{table}

The Thermal Group (Table \ref{ttab}) sensors are external temperature-sensitive resistors. The nominal range for the circuitry is 500� to 1500�, which translates into the given temperature ranges. It may be useful to calibrate the board using external fixed resistors near the limits of the range.

\end{document}
